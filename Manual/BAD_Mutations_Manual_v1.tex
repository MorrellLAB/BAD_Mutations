\documentclass[12pt]{article}
%   We will be using this term throughout, so we define a macro to make it easy
\newcommand{\BM}{\texttt{BAD\_Mutations} }
%   Load up packages here
\usepackage[right=1in, left=1in, top=1in, bottom=1in]{geometry}
\usepackage{palatino}
\usepackage{amsmath}
\usepackage{titlesec}
\usepackage{graphicx}
\usepackage{booktabs}
\usepackage{float}
\usepackage{xcolor}
\usepackage{fancyvrb}
\usepackage{fancyhdr}
\usepackage{enumitem}
\usepackage{multirow}
\usepackage[colorlinks=true, urlcolor=blue]{hyperref}
\pagestyle{fancy}
\fancyhead[RO]{\BM Documentation}
\fancyhead[LO]{Version 1 \\ December 14, 2015}
%   Set some lengths. No ident on new paragraphs, extra space after paragraph.
\setlength{\parindent}{0pt}
\setlength{\parskip}{\baselineskip}
%   For section headers, we want spacing above, but not below
\titlespacing\section{0pt}{12pt plus 2pt minus 2pt}{-5pt plus 2pt minus 2pt}
%   The header needs a few lines' worth of space to hold the text we want
\setlength{\headheight}{30pt}
\begin{document}

\section*{Overview}
\par \BM (\textbf{\underline{B}}LAST-\textbf{\underline{A}}ligned-\textbf{\underline{D}}eleterious?)
performs a likelihood ratio test (LRT) for the prediction of deleterious
variants. The package is comprised of Python and Bourne Again Shell (BASH) scripts.
The LRT is handled by a HYPHY script. \BM was written with Python 2 syntax, but
conversion to Python 3 is planned. \BM is designed to be run from the command
line. Running from an interactive Python environment is not recommended nor
supported.

\par \BM contains five major subcommands: \texttt{setup}, \texttt{fetch}, 
\texttt{align}, \texttt{predict}, and \texttt{compile}. Both \texttt{setup} and
\texttt{fetch} are meant to be run once, or very rarely. The \texttt{align}
subcommand generates phylogenetic trees and multiple sequence alignments for
input to the prediction scripts. The \texttt{predict} subcommand does the actual
variant effect prediction. More information about how to run \BM is available
in the ``Usage'' section.

\par Briefly, \BM predicts deleterious variants using a sequence constraint
approach. For a given query gene sequence and list of nonsynonmyous SNPs, a
multiple sequence alignment among orthologues is produced, and the given codons
are tested for conservation. Variants that alter a codon with a high degree of
conservation are inferred to be deleterious. More details on the procedure in
\BM is available in the ``Methods'' section.

\section*{Citation}
\par The model used to estimate codon conservation and predict which variants
are deleterious is reported in Chun and Fay (2009). The actual software package
is first used in Kono \textit{et al.} (In Prep.). \BM will have a formal
publication after the Kono \textit{et al.} manuscript is published.

\par \BM was primarily written by Thomas JY Kono and Paul J Hoffman. The HYPHY
script for estimating codon conservation was written by Justin C Fay. Testing
was performed by Chaochih Liu, Felipe Reyes, and Skylar Wyant.

\section*{Downloading}
\par \BM is distributed through a
\href{https://github.com/MorrellLAB/BAD_Mutations}{GitHub repository}. You can
use \href{https://git-scm.com/}{Git} to clone the repository, or download a ZIP
archive from GitHub.

\section*{Dependencies}
\par \BM is written to run in a UNIX-like environment. It has been successfully
run on both Apple OS X and GNU/Linux. It is not supported on Microsoft Windows.
It has not been tested on other variants of commercial UNIX.

\par \BM requires that the following software is installed and
available in your \texttt{\$PATH} or \texttt{sys.path} in Python:
\begin{itemize}[noitemsep]
    \item \href{https://www.gnu.org/software/bash/}{GNU Bash} $\geq$ 3.2
    \item \href{https://www.python.org/}{Python} $\geq$ 2.6.x
    \item \href{http://biopython.org/}{Biopython} 1.6x
    \item \href{https://code.google.com/p/argparse/}{argparse} (Python library) If using Python 2.6
    \item \href{https://blast.ncbi.nlm.nih.gov/Blast.cgi?PAGE_TYPE=BlastDocs&DOC_TYPE=Download}{BLAST+} $\geq$ 2.2.29
    \item \href{http://www.cs.utexas.edu/~phylo/software/pasta/}{PASTA}
    \item \href{http://hyphy.org/}{HyPhy} 2.2.x
    \item \href{http://curl.haxx.se/}{cURL}
\end{itemize}

\par We offer a script (\texttt{Shell\_Scripts/get\_dependencies.sh}) that will
fetch, compile, and install most of the dependencies for you. It will not
install GNU Bash or Python, as it assumes you have those available. The
dependencies will be installed locally, and so you must direct \BM to the paths
of the install programs in the configuration file.

\subsection*{Instructions for UMN MSI}
\par This section is specific to using \BM on the \href{http://msi.umn.edu/}
{University of Minnesoa Super Computing Institue} cluser. Our cluster uses the
\texttt{module} command to add and remove certain programs from the user's
environment. The following commands should be run for \BM on the cluster:
\begin{Verbatim}[frame=single, fontsize=\small, rulecolor=\color{gray}]
$ module load python2
$ module load biopython
$ module load ncbi_blast+
$ module load hyphy/2.2.6_smp
\end{Verbatim}
\par You will have to install texttt{PASTA} as its user manual instructs. cURL
should be available.

\section*{Input}
\par Input files should be plain text with UNIX line endings (LF). \BM takes a
FASTA file containing the query coding sequence, and a text file with the list
of codons to predict. The coding sequence does not have to start with ATG, but
it should be supplied in the 5$^{\prime}$ to 3$^{\prime}$ direction, and its
length should be a multiple of 3. The codons should be supplied as numerical
offsets with respect to the provided FASTA file, with counting starting from 1
and one codon per line. The substitutions file may optionally have a second
field with a SNP identifier.

\par There is no programmatic means of enforcing the consistency of
directionality between the FASTA file and the substitutions file. This means it
is possible to submit them in the reverse order, but keep in mind that the
coordinates must match in order for the predictions to be valid.

\par The FASTA input should look like this:
\begin{Verbatim}[frame=single, fontsize=\small, rulecolor=\color{gray}]
>Gene_1
ATGCCAGTGCAG...
...
\end{Verbatim}
And the substitutions file should look like this:
\begin{Verbatim}[frame=single, fontsize=\small, rulecolor=\color{gray}]
4   SNP_1
10  SNP_2
25  SNP_3
100 SNP_4
\end{Verbatim}

\par This pair of files would describe four nonsynonymous variants to predict
in a single coding sequence. The variants occur at residue numbers 4, 10, 25,
and 100 in the \textbf{amino acid} sequence, with the first residue being
treated as position 1. Their identifiers are \texttt{SNP\_1}, \texttt{SNP\_2},
\texttt{SNP\_3}, and \texttt{SNP\_4}, respectively. These may be any
non-whitespace text, and may be internal identifiers for bookkeeping, or
rs numbers, or some other SNP identification system.

\par Note that while the FASTA file contains \textbf{nucleotide} sequence, the
substitutions file contains positions in the \textbf{amino acid} sequence.
Support for nucleotide offsets is planned for a future version.

\section*{Output}
\par \BM will return a report on each queried position. Information returned
includes the number of species in the alignment, the homologous amino acid
state in each of the species, a constraint score, and the $p$-value that the
site is constrained across the evolutionary history of the species in the tree.

\par \textit{More on this later...}
\section*{Usage}
\subsection*{Basic Invocation}
\par \BM can be called from command line in a manner similar to UNIX programs.
You must either set the executable flag on the script \texttt{BAD\_Mutations.py},
or pass the script to the Python interpreter.
\begin{Verbatim}[frame=single, fontsize=\small, rulecolor=\color{gray}]
$ chmod +x BAD_Mutations.py
$ ./BAD_Mutations.py [Options] [Subcommand] [More Options ... ]
--OR--
$ python BAD_Mutations.py [Options] [Subcommand] [More Options ... ]
\end{Verbatim}
\par \BM offers three subcommands, \texttt{setup}, \texttt{fetch}, and
\texttt{predict}. They are summarized below.

\subsection*{Subcommands, Options, and Switches}
\par Note: \BM example command lines will be provided at the end of the setup,
predict and fetch sections below. 
\subsection*{General Options}
\par \BM takes the following general options:
\begin{table}[h]
    \centering
    \begin{tabular}{p{105pt} l p{260pt}}
    \toprule
    Option & Value & Description \\
    \midrule
    \texttt{-h} & NA & Show help message and exit. \\
    \midrule
    & 'DEBUG' & Be very verbose. Print all messages.\\ \cmidrule{2-3}
    & 'INFO' & Just print info, warning, and error messages. Useful for progress checking.\\ \cmidrule{2-3}
    \texttt{-v/--verbose} & 'WARNING' & Print warnings and errors. Default setting. \\ \cmidrule{2-3}
    & 'ERROR' & Only print error messages. \\ \cmidrule{2-3}
    & 'CRITICAL' & Print almost nothing. Critical failures only. \\
    \bottomrule
    \end{tabular}
\end{table}
\subsection*{The \texttt{setup} Subcommand}
\par The \texttt{setup} subcommand creates a configuration file that contains
paths to required executables, paths to data storage directories, BLAST search
parameters, alignment parameters, and prediction parameters. Running
\texttt{setup} is optional, but recommended as it makes standardizing across
genes and analyses much simpler. This subcommand can also download and compile
dependencies for \BM.

\par The \texttt{setup} subcommand takes the following options:
\begin{table}[H]
    \centering
    \begin{tabular}{p{105pt} l p{260pt}}
    \toprule
    Option & Value & Description \\
    \midrule
    \texttt{--list-species} & NA & Show all species databases available. \\
    \midrule
    \texttt{-c/--config} & [FILE] & Where to store the configuration file. Defaults to \texttt{LRTPredict\_Config.txt}. \\
    \midrule
    \texttt{-b/--base} & [DIR] & Directory to store the BLAST databases. Defaults to the current directory.\\
    \midrule
    \texttt{-d/--deps-dir} & [DIR] & Directory to download and store the dependencies. Defaults to current directory.\\
    \midrule
    \texttt{-t/--target} & [SP\_NAME] & Target species name. Must be one of the species (case sensitive) given by \texttt{--list-species}. This species will be excluded from the prediction pipeline to avoid reference bias. No default.\\
    \midrule
    \texttt{-e/--evalue} & [FLOAT] & E-value threshold for accepting TBLASTX hits as putative orthologues. Defaults to 0.05.\\
    \midrule
    \texttt{-m/--missing} & [INT] & Minimum number of gapped (missing) sites in the multiple species alignment (MSA) to be considered for prediction.\\
    \bottomrule
    \end{tabular}
\end{table}

\subsection*{The \texttt{fetch} Subcommand}
\par The \texttt{fetch} subcommand creates the necessary BLAST databases for
identifying orthologues. It will fetch gzipped CDS FASTA files from both
Phytozome 10 and Ensembl Plants, unzip them, and convert them into BLAST
databases. Fetching data from Phytozome requires a (free) account with
the \href{http://genome.jgi.doe.gov/}{JGI Genome Portal}. Note that not every
genome sequence in Phytozome is available to be used for this analysis. Check
the species info page on Phytozome for specific data usage policies.

\par The \texttt{fetch} subcommand accepts the following options:
\begin{table}[H]
    \centering
    \begin{tabular}{p{105pt} l p{260pt}}
    \toprule
    Option & Value & Description \\
    \midrule
    \texttt{-c/--config} & [FILE] & Path to configuration file. Defaults to \texttt{LRTPredict\_Config.txt}. \\
    \midrule
    \texttt{-b/--base}* & [DIR] & Directory to store the BLAST databases. Defaults to the current directory.\\
    \midrule
    \texttt{-u/--user} & [STR] & Username for JGI Genome Portal. Required.\\
    \midrule
    \texttt{-p/--password} & [STR] & Password for JGI Genome Portal. If not supplied on command line, will prompt user for the password.\\
    \midrule
    \texttt{--fetch-only} & NA & If supplied, do not convert CDS FASTA files into BLAST databases.\\
    \midrule
    \texttt{--convert-only} & NA & If supplied, only unzip and convert FASTA files into BLAST databases. Do not download.\\
    \bottomrule
    \end{tabular}
\end{table}
\par $^*$: If this value is supplied on the command line, it will override
the value set in the configuration file.

\subsection*{The \texttt{align} Subcommand}
\par The \texttt{align} subcommand will run BLAST to identify putative
orthologues against each species' CDS sequence database. The putative
orthologues are aligned with PASTA, and a phylogenetic tree is estimated from
the alignment.

\par The \texttt{align} subcommand accepts the following options:
\begin{table}[H]
    \centering
    \begin{tabular}{p{105pt} l p{260pt}}
    \toprule
    Option & Value & Description \\
    \midrule
    \texttt{-b/--base}* & [DIR] & Directory to store the BLAST databases. Defaults to the current directory.\\
    \midrule
    \texttt{-c/--config} & [FILE] & Path to configuration file. Defaults to \texttt{LRTPredict\_Config.txt}. \\
    \midrule
    \texttt{-e/--evalue}* & [FLOAT] & E-value threshold for accepting TBLASTX hits as putative orthologues. Defaults to 0.05.\\
    \midrule
    \texttt{-f/--fasta} & [FILE] & Path to FASTA file with query sequence. Required.\\
    \midrule
    \texttt{-o/--output} & [DIR] & Directory for output. Defaults to current directory.\\
    \bottomrule
    \end{tabular}
\end{table}
\par $^*$: If this value is supplied on the command line, it will override
the value set in the configuration file.

\subsection*{The \texttt{predict} Subcommand}
\par The \texttt{predict} subcommand will generate predictions for a list of
affected codons. It will run a BLAST search of the query sequence against each
CDS sequence that was downloaded with the \texttt{fetch} subcommand, pick the
likely homologous sequences, align them, and then use HyPhy to predict each
query codon.

\par The \texttt{predict} subcommand accepts the following options:
\begin{table}[H]
    \centering
    \begin{tabular}{p{105pt} l p{260pt}}
    \toprule
    Option & Value & Description \\
    \midrule
    \texttt{-a/--alignment} & [FILE] & Path to the multiple sequence alignment file. Required.\\
    \midrule
    \texttt{-c/--config} & [FILE] & Path to configuration file. Defaults to \texttt{LRTPredict\_Config.txt}. \\
    \midrule
    \texttt{-r/--tree} & [FILE] & Path to the phylogenetic tree. Required.\\
    \midrule
    \texttt{-s/--substitutions} & [FILE] & Path to substitutions file. Required\\
    \midrule
	\texttt{-o/--output} & [DIR] & Directory for output. Defaults to current directory.\\
    \bottomrule
    \end{tabular}
\end{table}
\par $^*$: If this value is supplied on the command line, it will override
the value set in the configuration file.

\subsection*{The \texttt{compile} Subcommand}
\par The \texttt{compile} subcommand will take an output directory containing
HyPhy output files, and produce a table with predictions for each variant. The
script will print $P$-values, but will not assess significance, as a suitable
significance threshold cannot be determined programmatically. This is left to
the user to interpret. This subcommand requires the output from another SNP
effect script, \href{https://raw.githubusercontent.com/TomJKono/Misc_Utils/master/SNP_Effect_Predictor.py}
{SNP\_Effect\_Predictor.py} (NOTE: requires the companion Python class defined
in \href{https://raw.githubusercontent.com/TomJKono/Misc_Utils/master/gff_parse.py}{gff\_parse.py}).

\par The \texttt{compile} subcommand accepts the following options:
\begin{table}[H]
    \centering
    \begin{tabular}{p{105pt} l p{260pt}}
    \toprule
    Option & Value & Description \\
    \midrule
    \texttt{-S/--long-subs} & [FILE] & Path to the SNP effect table. Required.\\
    \midrule
    \texttt{-p/--pred-dir} & [DIR] & Output directory from the \texttt{predict} subcommand. Required.\\
    \bottomrule
    \end{tabular}
\end{table}

\subsection*{Example Command Lines}
\par The following command line demonstrates the typical usage of \BM.

\par This command will set up the environment for predicting in barley
(\textit{Hordeum vulgare}), with very verbose output:
\begin{Verbatim}[frame=single, fontsize=\small, rulecolor=\color{gray}]
$ ./BAD_Mutations.py -v DEBUG \
                     setup \
                     -b /scratch/BAD_Mutations_Data \
                     -d /scratch/BAD_Mutations_Deps \
                     -t 'Hordeum_vulgare' \
                     -e 0.05 \
                     -m 0.2 \
                     -c BAD_Mutations_Config.txt 2> Setup.log
\end{Verbatim}

\par This command will download all of the necessary CDS sequences from both
Phytozome and Ensembl Plants and convert them into BLAST databases:
\begin{Verbatim}[frame=single, fontsize=\small, rulecolor=\color{gray}]
$ ./BAD_Mutations.py -v DEBUG \
                     fetch \
                     -c BAD_Mutations_Config.txt \
                     -u 'user@domain.com' \
                     -p 'ReallyGoodPassword123' 2> Fetch.log
\end{Verbatim}

\par And this command will predict the functional impact of variants listed in
\texttt{subs.txt} using \texttt{CoolGene.fasta} as a query:
\begin{Verbatim}[frame=single, fontsize=\small, rulecolor=\color{gray}]
$ ./BAD_Mutations.py -v DEBUG \
                     predict \
                     -c BAD_Mutations_Config.txt \
                     -f CoolGene.fasta \
                     -s subs.txt 2> CoolGene_Predictions.log
\end{Verbatim}

\section*{Configuration File Format}
\par\textbf{NOTE:} The configuration file format is under revision (in a new
git branch) and is planned to change soon. This section of the manual will be
updated when the new file format is deployed. The format will follow the
specifications used by the Python 
\href{https://docs.python.org/2/library/configparser.html}{ConfigParser}
module.

\par The configuration file is modeled after the configuration file of
STRUCTURE [Pritchard \textit{et al.}, (2000)]. A sample configuration file is
shown below:
\begin{Verbatim}[frame=single, fontsize=\small, rulecolor=\color{gray}]
// Generated by 'setup' at 2015-10-07 19:09:09.622228
#define BASE /scratch/BAD_Mutations_Data
#define TARGET_SPECIES hordeum_vulgare
#define EVAL_THRESHOLD 0.05
#define MISSING_THRESHOLD 10

// Program paths
#define BASH /usr/local/bin/bash
#define GZIP /usr/bin/gzip
#define SUM /usr/bin/sum
#define TBLASTX /usr/local/bin/tblastx
#define PASTA /usr/local/bin/run_pasta.py
#define HYPHY /usr/local/bin/HYPHYSP
\end{Verbatim}

\section*{Runtimes and Benchmarks}
\par By far, the slowest part of \BM is fetching CDS sequences and converting
them to BLAST databases. This may take up to several hours, depending on your
network and disk speeds. The databases and FASTA files take up approximately
4GB, as of October 2015. As more genomes are sequenced and annotated, this
figure will increase.

\par For a typical barley gene ($\approx$3000 bp), \BM can generate a
phylogenetic tree and multiple sequence alignment in approximately 5-10 minutes
on a desktop computer (Intel i7 2.8GHz). Note, however, that not every gene
will have every species represented in the alignment and tree. This is not a
problem for \BM.

\par Predictions are generated in two stages: a $\frac{dN}{dS}$ estimation
phase and a per-site prediction phase. The $\frac{dN}{dS}$ phase is slow; for
the same  $\approx$3000bp gene, the average time to estimate $\frac{dN}{dS}$
is 11319.5 CPU-seconds ($\approx$3 CPU-hours), with a standard deviation of
10803.9 CPU-seconds (also $\approx$3 CPU-hours). Per-site predictions are much
faster, with an average runtime of 73.9 CPU-seconds, and a standard deviation
of 67.8 CPU-seconds.

\par In all, BLAST searching and predicting for a single barley gene takes an
average of 3-4 CPU-hours to complete. The process is readily parallelizable on
a gene-by-gene basis. This makes processing a complete dataset consisting of
tens of thousands of genes feasible on a computing cluster.

\par Note, however, that runtimes will vary depending on the gene being
analyzed. Genes that are rapidly evolving will take longer in the BLAST search,
alignment, and prediction stages. The max amount of time it took for \BM to
calcuate $\frac{dN}{dS}$ was $\approx$46 CPU-hours, for instance.

\section*{Methods}
\par \BM uses TBLASTX to identify genes that are homologous to the query
sequence based on translated similarity. Hits that are above the user-supplied
E-value threshold are excluded. Once a list of orthlogues is identified,
\BM translates the sequences into amino acids, and aligns them with PASTA. A
phylogenetic tree of the species is also estimated from the alignment. The
alignment is then back-translated using the original nucleotide sequence hits
from their respective BLAST databases. This alignment is then supplied to the
prediction script, where the query codons are evaluated using HyPhy.

\par Evaluation of codons uses a likelihood ratio test (LRT) to give the
probability that a nonsynonymous SNP is deleterious. First, the ratio of the
local synonymous and nonsynonymous substitution rates ($\frac{dN}{dS}$) is
estimated from the gene alignment. Then, using those rates and the estimated
phylogenetic relationship among the sequences, the program tests the
likelihood of the queried codon evolving under selective constraint against the
likelihood of it evolving neutrally. For a full description of the statistical
model used, see 
\href{http://genome.cshlp.org/content/19/9/1553.abstract}{Chun and Fay (2009)}.

\par \BM makes several assumptions in its prediction pipeline. First, putative
orthologues identified with BLAST are assumed to have conserved function across
all of the species represented in the alignment. For some gene families,
particularly those involved in pathogen recognition and defense, this assumption
may not be true. Next, \BM assumes that the sequences identified as homologous
through sequence similarity are \textit{orthologous}. This assumption is
manifest in the multiple sequence alignment, as each site in the alignment is
then assumed to be orthologous. For gene families that are highly duplicated
(either proliferating, or due to a whole genome duplication event), this
assumption may also be violated. That is, sequences identified through BLAST
searching may be paralogous, and subject to a different mode of selection than
purifying selection.

\par As such, exercise caution when interpreting results from \BM.

\newpage
\section*{Data Sources}
\par As of October 2015, the following Angiosperm genomes (41) are available
for use in Ensembl and Phytozome:
\begin{table}[H]
\scriptsize
    \centering
    \begin{tabular}{l l l l l}
    \toprule
    Species & Common Name & Assembly Version & Annotation Version & Source\\
    \midrule
    \textit{Aegilops tauschii} & Goatgrass & ASM34733v1 & 1 & Ensembl Plants\\
    \textit{Aquilegia coerulea} & Columbine & 1.1 & 1.1 & Phytozome 10\\
    \textit{Arabidopsis lyrata} & Lyrate rockcress & 1.0 & 1.0 & Phytozome 10\\
    \textit{Arabidopsis thaliana} & Thale cress & TAIR10 & TAIR10 & Phytozome 10\\
    \textit{Boechera stricta} & Drummond's rockcress & 1.2 & 1.2 & Phytozome 10\\
    \textit{Brachypodium distachyon} & Purple false brome & 2.1 & 2.1 & Phytozome 10\\
    \textit{Brassica oleracea} & Cabbage & 2.1 & 2.1 & Ensembl Plants\\
    \textit{Brassica rapa} & Turnip mustard & FPsc 1.3 & 1 & Phytozome 10\\
    \textit{Capsella grandiflora}& -- & 1.1 & 1.1 & Phytozome 10\\
    \textit{Capsella rubella} & Red shepherd's purse & 1.0 & 1.0 & Phytozome 10\\
    \textit{Carica papaya} & Papaya & ASGPBv0.4 & ASGPBv0.4 & Phytozome 10\\
    \textit{Citrus clementina} & Clementine & 1.0 & clementine1.0 & Phytozome 10\\
    \textit{Citrus sinensis} & Sweet orange & 1.0 & orange1.1 & Phytozome 10\\
    \textit{Cucumis sativus} & Cucumber & 1.0 & 1.0 & Phytozome 10\\
    \textit{Eucalyptus grandis} & Eucalyptus & 2.0 & 2.0 & Phytozome 10\\
    \textit{Eutrema salsugineum} & Salt cress & 1.0 & 1.0 & Phytozome 10\\
    \textit{Fragaria vesca} & Strawberry & 1.1 & 1.1 & Phytozome 10\\
    \textit{Glycine max} & Soybean & a2 & a2.v1 & Phytozome 10\\
    \textit{Gossypium raimondii} & Cotton & 2.1 & 2.1 & Phytozome 10\\
    \textit{Hordeum vulgare} & Barley & 082214v1 & 1.0 & Ensembl Plants\\
    \textit{Leersia perrieri} & Cutgrass & 1.4 & 1.0 & Ensembl Plants\\
    \textit{Linum usitatissimum} & Flax & 1.0 & 1.0 & Phytozome 10\\
    \textit{Malus domestica} & Apple & 1.0 & 1.0 & Phytozome 10\\
    \textit{Manihot esculenta} & Cassava & 6.0 & 6.1 & Phytozome 10\\
    \textit{Medicago truncatula} & Barrel medic & Mt4.0 & Mt4.0v1 & Phytozome 10\\
    \textit{Mimulus guttatus} & Monkey flower & 2.0 & 2.0 & Phytozome 10\\
    \textit{Musa acuminata} & Banana & MA1 & MA1 & Ensembl Plants\\
    \textit{Oryza sativa} & Asian rice & IRGSP-1.0 & 7.0 & Phytozome 10\\
    \textit{Panicum virgatum} & Switchgrass & 1.0 & 1.1 & Phytozome 10\\
    \textit{Phaseolus vulgaris} & Common bean & 1.0 & 1.0 & Phytozome 10\\
    \textit{Populus trichocarpa} & Western poplar & 3.0 & 3.0 & Phytozome 10\\
    \textit{Prunus persica} & Peach & 2.0 & 2.1 & Phytozome 10\\
    \textit{Ricinus communis} & Castor bean & 0.1 & 0.1 & Phytozome 10\\
    \textit{Setaria italica} & Foxtail millet & 2.0 & 2.1 & Phytozome 10\\
    \textit{Solanum lycopersicum} & Tomato & SL2.50 & iTAG2.3 & Phytozome 10\\
    \textit{Solanum tuberosum} & Potato & 3\_2.1.10 & 3.4 & Phytozome 10\\
    \textit{Sorghum bicolor} & Milo & 2.0 & 2.1 & Phytozome 10\\
    \textit{Theobroma cacao} & Cacao & 1.0 & 1.0 & Phytozome 10\\
    \textit{Triticum urartu} & Red wild einkorn & ASM34745v1 & 1 & Ensembl Plants\\
    \textit{Vitis vinifera} & Grape & Genoscope.12X & Genoscope.12X & Phytozome 10\\
    \textit{Zea mays} & Maize & 6a & 6a & Phytozome 10\\
    \bottomrule
    \end{tabular}
\end{table}
\end{document}
